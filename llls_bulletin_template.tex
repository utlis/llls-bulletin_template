%------------------------------------------------------------
%生涯学習基盤経営 TeXテンプレート Ver. 2.1
%arranged by Fukuji IMAI
%改訂内容:
%・英語タイトルの誤記,氏名および所属見本の改訂を行った。
%------------------------------------------------------------
%生涯学習基盤経営 TeXテンプレート Ver. 2
%arranged by Fukuji IMAI
%改訂内容:
%・スタイルファイルVersion2を参照するようにした。
%・(スタイルファイルVersion2への対応)副題に対応するためにjtitleの引数を2つ取るようにした。
%・(スタイルファイルVersion2への対応)英語要約の行間広げ,字下げ対応のため,eabstract環境を新設。
%------------------------------------------------------------
\documentclass[b5paper,10pt,twocolumn,tombow]{jarticle}
\usepackage[size=a4paper,priority=high]{bxpapersize}
\usepackage{llls}
\usepackage{graphicx}

\begin{document}
\twocolumn[
\voltitle{41}{2016}

\begin{center}
  \jtitle{『生涯学習基盤経営』における論文等の執筆マニュアル}{---\LaTeX{}
  バージョン---}
\end{center}

\begin{authors}
  \name{1}{本郷弥生}
  \name{2}{東大太郎}
\end{authors}
\begin{affiliation}
  \aff{1}{東京大学大学院教育学研究科}
  \aff{2}{生涯学習基盤経営学会}
\end{affiliation}

\begin{center}
  \begin{abstract}
    この文章は『生涯学習基盤経営』の論文等(論文,研究ノート,資料)のレイアウ
    トを説明した文章です。この文章自体が執筆要領を兼ねておりますので,この
    ファイルをそのまま使用して論文を作成することを推奨します。
  \end{abstract}
  \begin{keyword}
    キーワード: 生涯学習基盤経営,執筆要領,レイアウト
  \end{keyword}
\end{center}
\initialize
]
\tableofcontents{}
\bigskip{}

\section{はじめに}
この文章は『生涯学習基盤経営』の論文等(論文,研究ノート,資料)のレイアウ
トを説明した文章です。この文章自体が執筆要領を兼ねております。このファイ
ルをそのまま使用して論文を作成することを推奨します。本稿をよくお読みいた
だき,原稿を作成して下さい。


なお,このファイルでは\LaTeX{}のレイアウトが記載されています。
Microsoft Word\textsuperscript{\textregistered}(以下,Word)の
レイアウト規定を確認されたい場合には,Wordのテンプレートファイルをご参照
下さい。なお,本要領は情報メディア学会の執筆要項ファイルを参照しました
\footnote{情報メディア学会. 『情報メディア研究』への各種原稿の投稿につい
て. 入手先URL:\url{http://www.jsims.jp/toko.html}, (2008-10-27参照)}。

\LaTeX{}のテンプレートファイルを使用する場合にはスタイルファイルが必要です。
テンプレートファイルのアーカイブにスタイルファイルが同封してあります。テ
ンプレートファイルと同じディレクトリに設置するか,\LaTeX{}のスタイルファイ
ル格納場所(環境によって異なります)に設置して利用してください。現在のス
タイルファイル名は\texttt{llls.sty}です。

\LaTeX{}の原稿執筆には環境,並びにコマンドを使用します。本テンプレートファ
イルにおいて,環境とは\verb|\begin{**}|で始まり,\verb|\end{**}|
で閉じる命令のことを指し,コマンドとは\verb|\textit{**}|のように,引
数を伴う命令のことを指します。またコマンドによっては引数を複数取ることが
あります。例えば紀要の号数と出版年を記入する\verb|\voltitle|コマン
ドは
\begin{verbatim}
\voltitle{35}{2010}
\end{verbatim}
と2つの引数を取ります。本マニュアルでは,この2つの引数をコマンド
の左側からそれぞれ,第1引数,第2引数と呼ぶこととします。

\section{原稿について}
A4版を使用し,原則として投稿段階から\LaTeX{}あるいはMicrosoft社のWordによっ
て,原稿を作成してください。作成の際には,必ずテンプレートファイルに基づ
き,印刷原稿にトンボが適切に出力されるように,作成して下さい。(紀要
はB5サイズで印刷されます)

\section{全体的なレイアウトについて}
\subsection{段組および字数・行数}

本文に先行するヘッダーの部分(論文等タイ
トル,著者氏名,要約)および,英文要約については一段組で,本文については
二段組1行22字で,縦は44行程度で設定してください。\LaTeX{}の場合,本テンプレー
トの使用により,同等の出力が得られます。


\subsection{文字サイズと字体}
\LaTeX{}の文字サイズについては,スタイルファイルである\texttt{llls.sty}
によって定義されていますので,著者側で文字サイ
ズの変更等は行わないで下さい。書体については,日本語は,論
文等タイトルおよび章,節のタイトルでは,ゴシック体を使用し,それ以外につ
いては明朝体を使用してください。英語についてはローマン体を使用して下さい。
(デフォルトの設定でこれらの字体が使われますので通常,特に意識する必
要はありません。)

\subsection{句読点}
句読点は,日本語では,句点として``。'',読点として``,''をそれぞれ
用います。英語については半角のピリオドと半角スペース,読点は半角のカンマ並
びに半角スペースを使用してください。

\section{各構成要素別のレイアウトについて}
\subsection{論文の構成要素}
日本語論文等の構成要素,並びに順序は次の通りです:
\begin{enumerate}
  \item 号数,出版年
  \item 和文タイトル(およびサブタイトル)
  \item 和文著者名
  \item 和文著者所属
  \item 和文要約(300--400字)
  \item 和文キーワード(3語程度を著者で付与)
  \item 目次
  \item 本文(文末注を含む)
  \item 文章末の要約情報
  \begin{enumerate}
    \item 英文タイトル(およびサブタイトル)
    \item 英文著者名
    \item 英文著者所属
    \item 英文要約(100--150 words)
    \item 英文キーワード(3語程度を著者で付与)
  \end{enumerate}
\end{enumerate}
英語論文等の構成要素は次の通りです。
\begin{enumerate}
  \item 号数,出版年
  \item 英文タイトル(およびサブタイトル)
  \item 英文著者名
  \item 英文著者所属
  \item 英文要約(100--150 words)
  \item 英文キーワード(3語程度を著者で付与)
  \item 目次
  \item 本文(文末注を含む)
  \item 文章末の要約情報
  \begin{enumerate}
    \item 和文タイトル(およびサブタイトル)
    \item 和文著者名
    \item 和文著者所属
    \item 和文要約(300--400字)
    \item 和文キーワード(3語程度を著者で付与)
  \end{enumerate}
\end{enumerate}
\subsection{号数,出版年}
紀要の号数並びに出版年を記入します。

\verb|\voltitle|コマンドの中に,投稿する号に合わせて``号数''と``出版
年''を記入します。コマンドの第1引数には``号数''を半角のアラビア数字で,
第2引数には``出版年''を西暦かつ半角のアラビア数字で記入してください。
\begin{verbatim}
\voltitle{NUMBER}{YEAR}
\end{verbatim}


例えば,2010年度の35号であれば,
\begin{verbatim}
\voltitle{35}{2010}
\end{verbatim}
と入力してください。

\subsection{和文タイトル}
論文のタイトルを日本語で記入します。
\verb|\jtitle{}|コマンドの第1引数にタイトル(主題)を記入してくだ
さい。サブタイトル(副題)を記入する場合は第2引数に記入します。サブタイ
トルを付与しない場合には第2引数は``\{\}''の状態(空の状態)にします。
\begin{verbatim}
\jtitle{タイトル}{サブタイトル}
\end{verbatim}

\subsection{和文著者名}
論題に対応する和文著者名を記入します。\texttt{authors}環境の
中に,\verb|\name|コマンドが含まれています。コマンドの引数は第1引数
に何番目の著者かを記入し,第2引数には著者名を記入します。ま
た,\verb|\name|コマンドを書いた数だけ,著者名と対応する記号が付与される
ようになっています。姓と名のあいだはスペースを空けずに入力してください。

テンプレートファイル上では,著者が2名存在する場合のサンプルを記入していま
す。著者が1名の場合には,\verb|\name|コマンドの\verb|\name{2}{}|以降を削除し
\verb|\name{1}{}|の部分だけを使用してください。著者が3名以上いる
場合には,\verb|\name|コマンドを著者の人数分増やして,順序に応じて
\verb|\name|コマンドの最初の引数を変更して使用してください。例え
ば,3人の場合には以下のようになります。
\begin{verbatim}
\begin{authors}
  \name{1}{本郷弥生}
  \name{2}{東大太郎}
  \name{3}{柏駒場}
\end{authors}
\end{verbatim}
本テンプレートファイルでは著者名を5つまで記入することができます。著者
名欄が6つ以上必要な場合は,スタイルファイルの改変が必要ですので,事
前にご連絡下さい。


\subsection{和文著者所属}
著者に対応する和文所属を記入します。\verb|\affiliation|環境の中に,
\verb|\aff|コマンドが含まれています。コマンドの引数は第1引数に所属
の記述順(\verb|\name|コマンドの第1引数と対応させてください)を記入
し,第2引数には所属の英文名を記入します。


テンプレートファイル上では,著者が2名存在する場合のサンプルを記入しています。も
し,著者が1名の場合には,\verb|\aff|コマンドの\verb|{2}{}|以降を削除し
\verb|\name{1}{}|の部分だけを使用してください。第3著者以降がいる
場合には,\verb|\aff|コマンドを著者の人数分増やして,順序に応じて
\verb|\aff|コマンドの最初の引数を変更して使用してください。例え
ば,3人の場合には以下のようになります。

\begin{verbatim}
\begin{affiliation}
  \aff{1}{東京大学大学院教育学研究科}
  \aff{2}{生涯学習基盤経営学会}
  \aff{3}{その他組織}
\end{eaffiliation}
\end{verbatim}

本テンプレートファイルでは所属を5つ
まで記入することができます。所属欄が6つ以上必要な場合は,スタイルファ
イルの改変が必要ですので,事前にご連絡下さい。


\subsection{和文要約}
論文等の要約を記入します。\texttt{abstract}環境の間に\textbf{300字から400字で},
記載してください。要約の中では任意の改行コマンド
(\verb|\\|)の使用や,段落変えは行わないようにしてください。
例えば,次のように入力します。
\begin{verbatim}
\begin{abstract}
  この文章は『生涯学習基盤経営』の論文等(
  論文,研究ノート,資料)のレイアウトを説
  明した文章です。この文章自体が執筆要領を
  兼ねておりますので,このファイルをそのま
  ま使用して論文を作成することを推奨します。
\end{abstract}
\end{verbatim}


\subsection{キーワード}
論文等に付与するキーワードを\textbf{3語程度}記入します。\texttt{keyword}環境の中に
の間に``キーワード:''と記載してから,入力してください。キーワード間の区
切りは半角のカンマと半角のスペース1文字入力してください。
\begin{verbatim}
\begin{keyword}
  キーワード: 生涯学習基盤経営, 執筆要領, レイアウト
\end{keyword}
\end{verbatim}
なお,巻末の英文要約作成エリアにも\texttt{keyword}環境がありますが,書式が異なり
ますので,混同しないように注意してください。
\subsection{目次}
目次については,\verb|\tableofcontents{}|コマンドにより,自動的に
作成されます。
\subsection{本文}
本文は,\verb|\tableofcontents|コマンドの直後にある
\verb|\bigskip|コマンド
の次の行から入力していってください。
\subsubsection{章,節,項のタイトル}
本文中の章,節,項はそれぞれ\verb|\section{}|,\verb|\subsection{}|,
\verb|\subsubsection{}|コマンドを使用してください。

\subsubsection{文章・表記など}
文章は原則として常用漢字と現代仮名遣いを用いてください。なお,以下の記号
については特定の用法で使ってください。
\begin{table}[h!]
  \center
  \small
  \begin{tabular}{cl}
    \toprule
    記号 & 用法 \\
    \midrule
    ( ) & 説明・その他付加的に記述する事柄 \\
    `` '' & 引用箇所の表示 \\
    \textit{Italic} & 文中における欧文の書名・誌名 \\
    『 』 & 文中における和文の書名・誌名 \\
    \bottomrule
  \end{tabular}
  \caption{特定の記号の用法}
\end{table}
\normalsize


\subsubsection{図,表}
図,表は本文中の適当な箇所に挿入してください。なお,図表は一つあたり400字あるい
は200ワード換算で数え,原稿の分量に加算します。

\subsubsection*{(1)図について}
\paragraph*{小さな図の場合}
下記のように,\texttt{figure}環境,\texttt{includegraphics}コマンド,\texttt{caption}コマンドを使用
し,図と図の下部に対応するcaptionが設定されるようにしてください。

\begin{verbatim}
\begin{figure}[tb]
  \centering
  \includegraphics[width=120pt]{test.eps}
  \caption{小さな図の例}
\end{figure}
\end{verbatim}

\begin{figure}[tb]
  \centering
  \includegraphics[width=120pt]{test.eps}
  \caption{小さな図の例}
\end{figure}

\paragraph*{大きな図の場合}
段をまたぐような大きな図を使用したい場合は,「小さな図の場合」で用いた
\texttt{figure}環境の代わりに,\texttt{figure*}環境を利
用します。その上で,\verb|\includegraphics|コマンド,\verb|\caption|コ
マンドを使用し,図と図の下部に対応するcaptionが設定されるようにしてください。

\begin{verbatim}
\begin{figure*}[tb]
  \centering
  \includegraphics[width=200pt]{test.eps}
  \caption{大きな図の例}
\end{figure*}
\end{verbatim}

\begin{figure*}[tb]
  \centering
  \includegraphics[width=200pt]{test.eps}
  \caption{大きな図の例}
\end{figure*}


\subsubsection*{(2)表について}
\paragraph*{小さな表の場合}
下記のように,\texttt{table}環境と\texttt{tabular}環境,\verb|\caption|コマンドを使用し,表の
下部に対応するcaptionを設定してください。

\begin{verbatim}
\begin{table}[tb]
  \centering
  \begin{tabular}{cc}
    \toprule
    日本語 & 英語 \\
    \midrule
    教授 & professor \\
    大学院生 & graduate student \\
    学部生 & undergraduate \\
    \bottomrule
  \end{tabular}
  \caption{小さな表の例}
\end{table}
\end{verbatim}

\begin{table}[tb]
  \centering
  \begin{tabular}{cc}
    \toprule
    日本語 & 英語 \\
    \midrule
    教授 & professor \\
    大学院生 & graduate student \\
    学部生 & undergraduate \\
    \bottomrule
  \end{tabular}
  \caption{小さな表の例}
\end{table}

\paragraph*{大きな表の場合}
段をまたぐような大きな表を使用したい場合は,「小さな表の場合」で用いた
\texttt{table}環境の代わりに,\texttt{table*}環境を利
用します。その上で,\texttt{tabular}環境,\verb|\caption|コ
マンドを使用し,表と表の下部に対応するcaptionが設定されるようにしてください。

\begin{verbatim}
\begin{table*}[tb]
  \centering
  \begin{tabular}{cc}
    \toprule
    日本語 & 英語 \\
    \midrule
    教授 & professor \\
    大学院生 & graduate student \\
    学部生 & undergraduate \\
    \bottomrule
  \end{tabular}
  \caption{大きな表の例}
\end{table*}
\end{verbatim}

\begin{table*}[tb]
  \centering
  \begin{tabular}{cc}
    \toprule
    日本語 & 英語 \\
    \midrule
    教授 & professor \\
    大学院生 & graduate student \\
    学部生 & undergraduate \\
    \bottomrule
  \end{tabular}
  \caption{大きな表の例}
\end{table*}


\subsubsection{引用}
本文中に他の文献からの引用を含める場合には,引用符`` ''を
用いて記述してください。
1文(ないし数文)からなるような比較的短い文章を引用する場合,\texttt{quote}環境を用いることも可能です。
引用文が長く,独立した段落として表示する必要がある場合には,\texttt{quotation}環境を用いてください。
いずれの引用方法でも,末尾に
\verb|\footnote|コマンドで,出典を記載してください。出典の記述方法に
ついては,\ref{123950_27Oct08}の「注・文献」を参考にしてください。
なお,引用した文書に番号を振る場合,閉引用符の後ろに付してください
(例えば,``~である''X。)。
出典の記述方法については,\ref{123950_27Oct08}の「注・文献」を参考にしてください。

\noindent{}\textbf{例1(引用符使用)}\\
澤田昭夫はよい論文について``よい論文は統一unity,連関coherence,
展開developmentにおいて優れた論文あるいは明確性clarityにおいて優れた論
文''\footnote{澤田昭夫. 『論文のレトリック』 講談社学術文庫, 講談社,1983, 330p. 引用は
p.~19}であると述べている。

\noindent{}\textbf{例2(quotation環境使用)}\\
澤田は論文執筆の際に次のようなことが重要だと述べている。
\begin{quotation}
  論文書きでもっとも大切なのは,問を疑問文の形で切り出すことで,それがレ
  トリックで言う発見・構想です。もっとも大切だというのは,それができれば
  つまり全体を貫く主な問が何であるかを確定することができれば,論文の首尾
  一貫性,統一性を保証する基本的条件が整ったことになるからです。

  そのつぎに大切なのは,論文の構成,材料の配置です。その際,肝に銘じなけれ
  ばならないのは,構成・配置の大原則は起承転結ではなく,序・本・結(序と
  本論と結び)だということです\footnote{\textit{Ibid.} p.~74}。
\end{quotation}


\subsection{注・文献} \label{123950_27Oct08}
本テンプレートファイルでは,注および文献については\verb|\footnote|コ
マンドを
使用し,全て文末に記載します。ページごとの注(脚注)及び独立した節としての参考文献リスト(\texttt{bibitem},\texttt{bibtex}を使った記述)は用いません。
注・文献における文献の記載方法は次の例に従ってください\footnote{特に、ページ番号には En dash ( -- ; \LaTeX 上ではハイフン2個 \texttt{ --} で出力できる) を必ず用いるようにしましょう。}。
(なお,記載方
法の中で,\{\}で囲まれた項目の記述は任意です。例えば,図書の場合は``版
表示'',``出版地'',``総ページ数''が任意の記述項目です。)

\subsubsection{図書の場合}
\noindent{}和:著編者名 『書名』 \{版表示,\} \{出版地,\} 出版社, 出版年, \{総
\bigskip
ページ数,\} 当該部分のページ.\\
洋:author. \textit{title}. \{edition,\} place of publication,
publisher, year, \{total page,\} page.
\begin{quote}
 近藤二郎 『社会科学のための数学入門』 東京経済新報社, 1973,
 p. 37--40.

 Barzun, Jacques and Graff, H. F. \textit{The Modern Researcher.}
 Rev. ed., \\New York, Harcourt, 1970, p. 165.
\end{quote}


\subsubsection{翻訳書の場合}
\noindent{}和:著編者名 『書名』[原書名(イタリックで記載) \{版表示,\}
\{出版地, \} 出版社, 出版年,] 翻訳者名, 出版社, 出版年, \{総
\bigskip
ページ数,\} 当該部分のページ.\\
洋:author. \textit{title of translation}. [\textit{original
title}. \{edition,\} place of publication, publisher, year,] tr. by
translator, place of publication, publisher, year, \{total page,\} page.

\begin{quote}
Varles, Jana ed. 『情報の要求と探索』 [\textit{Information Seeking:
 Basing Services on User's Behaviors.} North Calolina,
 McFarland \& Company, 1987] 池谷のぞみ, 市古健次, 白石英理子, 田村俊作訳,
 勁草書房, 1993, p. 10.

Schneider, Georg. \textit{Theory and History of Bibliography.}
 [\textit{Handbuch der Bibliographie.} Aufl., Berlin, Knopt, 1978,]
 tr. by R. R. Shaw, New York, Columbia University Press, 1934, p. 14--15.
\end{quote}

\subsubsection{編集書の一部(図書形態の論文集の一論文を含む)の場合}
\noindent{}和:当該部分の執筆者名 ``当該部分の題名'' $<$編者名 『書名』 \{版表示,\} \{出版地,\} 出版社, 出版年$>$ \{総
\bigskip
ページ数,\} 当該部分のページ.\\
洋:author. ``\textit{title},'' in editor. \textit{book title}, \{edition,\} place of publication,
publisher, year, \{total page,\} page.

\begin{quote}
宮坂広作 ``余暇と社会教育'' $<$碓井正久編著 『社会教育』 第一法規,
 1970$>$ p. 201--203.

Groom, Geofrey. ``\textit{Bibliography of older material},'' in Garvin,
 L. H. ed. \textit{Printed Reference Material.} 2nd ed., London, Library
 Association, 1984, p. 456--501.
\end{quote}


\subsubsection{逐次刊行物掲載記事(雑誌論文を含む)の場合}
\noindent{}和:執筆者名 ``論題名'' 『掲載逐次刊行物名』 vol. XX,
\{no. XX,\} 発行年\{月\}, 当該部分のページ.\\
洋:author. ``title,'' \textit{name of periodical}, vol. XX,
\{no. xx,\} year \{month\}, page.


\begin{quote}
小野寺夏生 ``Bibliostatistics'': 情報現象の統計学的説明'' 『情報管理』
 vol. 21, no. 10, 1979, p. 782--802.

小野寺夏生, 中井浩 ``単純なモデルからのZipfの法則の導入'' 『情報科学技術
 研究集会論文集』 vol. 33, no. 3, 1977, p. 129--138.

Brookes, Bertram C. ``Theory of the Bradford Laws,'' \textit{Journal of
 Documentation,} vol. 33, no. 3, 1977, p. 180--209.

Nelson, Micheal J. and Tague, Jean M. ``Sprit Size-Rank Models for the
 Distribution of Index Terms,'' \textit{Journal of the American Society
 for Information Science,} vol. 36, no. 5, 1985, p. 283--296.
\end{quote}

\subsubsection{Web上のリソースについて}
Web上のリソースについては,書誌情報の最後に入手先URLとアクセスした日付を記入します。
書式は``入手先URL: http://www.p.u-tokyo.ac.jp/(アクセス日: 2008-10-27 )''または
(``available from http://www.p.u-tokyo.ac.jp/ (accessed date: 2008-10-27)'')を記入します。
それ以外の項目は図書並びに逐次刊行物掲載記事の規定に準じ,入手先の情報から明らかである項目を記述します。
URLの記述には\verb|\url|コマンドを使ってください。

\begin{quote}
  情報メディア学会. 『『情報メディア研究』への各種原稿の投稿について』 入手
  先URL: \url{http://www.jsims.jp/toko.html} (アクセス日: 2008-10-27)
\end{quote}

\subsubsection{書誌事項の記載における省略語の使用について}

同一文献を二度以上引用する場合は,和文文献,欧文文献どちらの場合でも,
\textit{op. cit.}(前掲文献の意)\textit{Ibid.}(上掲文献の意)を用います。
こちらはイタリック体(斜字体)の代わりに下線を用いても良いこととします。
な\textit{op. cit.}を用いる場合,name,\textit{op. cit.}, (year,) p. Xのように記載します。
\textit{Ibid.}を用いる場合については,本文末注3の使用例を参照してください。


\subsection{文章末の要約情報}
論文等の最後には,改ページを行った後,本文を記述した言語以外のタイトル,
著者名,所属,要約,キーワード情報を記載します。本文を日本語で作成
した場合は,英文の情報を,本文を英文で記述した場合は,日本語の情報を記載
することになります。

\LaTeX{}の場合には,テンプレートファイルの最後に英文要約作成エリ
アが用意されています。テンプレートファイル上では,

\small
\begin{verbatim}
%--------------------------------------
%英文要約作成用エリア(\eauthors,
%\eaffiliation,\etitle,
%\abstract, \keywordの欄を書き換えて使う)
%--------------------------------------
\end{verbatim}
\normalsize

と記述されている部分から下のコードが該当します。

\subsubsection{作成エリア上の必須環境,必須コマンド}
まず,作成エリア直後のおよび末尾の以下のコマンドは,文章整形のために必要
な記述ですので,削除しないでください。
\begin{verbatim}
\newpage
\twocolumn[
  \begin{center}
    ......
  \end{center}
]
\end{verbatim}


\subsubsection{英文タイトル}
論題に対応する英文タイトルを記入します。\verb|\etitle|コマンドの引数
に対応するタイトルを記入してください。
文中の冠詞と前置詞と接続詞を除いて,各語の先頭は大文字とします。


\begin{verbatim}
\etitle{Manual of Writing a Manuscript
for Articles, Research Notes, and
Materials in \textit{Studies in Lifelong
Learning Infrastructure Management}}
\end{verbatim}

\subsubsection{英文著者名}
英語で著者名を名姓の順で記入します(名は頭文字のみ大文字,姓は全て大文字)。\verb|\eauthors|環境の
中に,\verb|\name|コマンドが含まれています。コマンドの引数は第1引数
に何番目の著者かを記入し,第2引数には著者名を記入します。ま
た,\verb|\name|コマンドを書いた数だけ,著者名と対応する記号が付与される
ようになっています。


テン
プレートファイル上では,著者が2名存在する場合のサンプルを記入していま
す。著者が1名の場合には,\verb|\name|コマンドの\verb|{2}{}|以降を削除し
\verb|\name{1}{}|の部分だけを使用してください。著者が3名以上いる
場合には,\verb|\name|コマンドを著者の人数分増やして,順序に応じて
\verb|\name|コマンドの最初の引数を変更して使用してください。例え
ば,3人の場合には以下のようになります。
\begin{verbatim}
\begin{eauthors}
  \name{1}{Yayoi HONGO}
  \name{2}{Tarou TOUDAI}
  \name{3}{Komaba KASHIWA}
\end{eauthors}
\end{verbatim}
本テンプレートファイルでは著者名を5つまで記入することができます。著者
名欄が6つ以上必要な場合は,スタイルファイルの改変が必要ですので,事
前にご連絡下さい。


\subsubsection{英文所属}
著者に対応する英文所属を記入します。\verb|\eaffiliation|環境の中に,
\verb|\eaff|コマンドが含まれています。コマンドの引数は第1引数に所属
の記述順(\verb|\name|コマンドの第1引数と対応させてください)を記入
し,第2引数には所属の英文名を記入します。


テン
プレートファイル上では,著者が2名存在する場合のサンプルを記入しています。も
し,著者が1名の場合には,\verb|\eaff|コマンドの\verb|{2}{}|以降を削除し
\verb|\name{1}{}|の部分だけを使用してください。第3著者以降がいる
場合には,\verb|\eaff|コマンドを著者の人数分増やして,順序に応じて
\verb|\eaff|コマンドの最初の引数を変更して使用してください。例え
ば,3人の場合には以下のようになります。

\begin{verbatim}
\begin{eaffiliation}
  \eaff{1}{Graduate School of Education,
  the University of Tokyo}
  \eaff{2}{Society for Lifelong Learning
Infrastructure Management}
  \eaff{3}{Another Organization}
\end{eaffiliation}
\end{verbatim}
本テンプレートファイルでは所属を5つ
まで記入することができます。所属欄が6つ以上必要な場合は,スタイルファ
イルの改変が必要ですので,事前にご連絡下さい。

なお,「東京大学大学院教育学研究科」の英文所属は``Graduate School of
Education, the University of
Tokyo''で統一するようお願いいたします。

\subsubsection{英文要約}
論文の内容に対する要約を英文で記入します。下記のサンプルのように,
\texttt{eabstract}環境の中に要約を\textbf{100語から150語で}記入してください。


\begin{verbatim}
\begin{eabstract}
  The paper describes style and layout of
  manuscripts in the Studies in \textit{
  Lifelong Learning Infrastructure
  Management}.
  You can use directly this
  file when you make your manuscript.
\end{eabstract}
\end{verbatim}


\subsubsection{英文キーワード}
論文の内容に対するキーワードを記入します。下記のように,
\texttt{keyword}環境の中に適切なキーワードを英語で\textbf{3つ程度}記入して
ください。文中の冠詞と前置詞と接続詞を除いて,各語の先頭は大文字とします。

\begin{verbatim}
\begin{keyword}
  Keywords: Studies in Lifelong Learning
  Infrastructure Management,
  Script Manual, Layout
\end{keyword}
\end{verbatim}


%--------------------------------------
%注エリアの作成用コマンド(消さないこと!)
%--------------------------------------
\begingroup \parindent 0ex \parskip 0.5ex \def\ennotesize{\normalsize}
\theendnotes \endgroup
%--------------------------------------
%英文要約作成用エリア(\eauthors,
%\eaffiliation,\etitle,
%\abstract, \keywordの欄を書き換えて使う)
%--------------------------------------
\newpage
\twocolumn[
\begin{center}
  \etitle{Manual of Writing a Manuscript for Articles, Research Notes, and
  Materials in \textit{Studies in Lifelong Learning Infrastructure Management}}
  \begin{eauthors}
    \name{1}{Yayoi HONGO} \name{2}{Tarou TOUDAI}
  \end{eauthors}
  \begin{eaffiliation}
    \eaff{1}{Graduate School of Education, the University of Tokyo}
    \eaff{2}{Society for Lifelong Learning Infrastructure Management}
  \end{eaffiliation}
  \begin{eabstract}
    The paper describes style and layout of manuscripts in \textit{the Studies in
    Lifelong Learning Infrastructure Management}. You can use
    directly this
    file when you make your manuscript.
  \end{eabstract}
  \begin{keyword}
    Keywords: Studies in Lifelong Learning Infrastructure Management, Script Manual, Layout
  \end{keyword}
\end{center}
]


\end{document}
